\section{Trigonometrie}

\begin{sectionbox}
	
	\subsection{Allgemeines Dreieck}
	Winkelsumme im Dreick
	
	\begin{tikzpicture}

		% Define the points of the triangle
		\coordinate (A) at (0,0);
		\coordinate (B) at (4,0);
		\coordinate (C) at (3.0,2.0);
		
		% Draw the triangle
		\draw [thick] (A) -- (B) -- (C) -- cycle;
		
		% Mark the angles
		\draw (A) ++(0.4,0) arc[start angle=0,end angle=33.7,radius=0.4];
		\node[above right= 0.1cm and 0.5cm of A] {$\alpha$};
		

		\centerarc[](B)(180.0:116.6:0.4)
		\node[above left= 0.1cm and 0.5cm of B] {$\beta$};

		\centerarc[](C)(213.7:296.6:0.4)
		\node[below left= 0.5cm and 0.1cm of C] {$\gamma$};

		% Label the vertices
		\node[below left] at (A) {A};
		\node[below right] at (C) {C};
		\node[above right] at (B) {B};

		% Label the sides
		\node (a) at ($(B)!0.5!(C)$) [anchor=west] {$a$};
		\node (b) at ($(A)!0.5!(C)$) [anchor=south east] {$b$};
		\node (c) at ($(A)!0.5!(B)$) [anchor=north] {$c$};
		
	\end{tikzpicture}
	
	\subsection{Rechtwinkliges Dreieck}
	AK: Ankathete \\
	GK: Gegenkathete \\
	HT: Hypotenuse \\
	
	\begin{tikzpicture}
		
		% Define the points A, B (endpoints of the diameter)
		\coordinate (A) at (0, 0);
		\coordinate (B) at (5, 0);
		
		% Define the midpoint M
		\coordinate (M) at ($(A)!0.5!(B)$);
		
		% Define point C on the semicircle
		\coordinate (C) at ($(M) + (50:2.5cm)$);  % Angle 50 degrees from the horizontal
		
		% Draw the semicircle
		\draw[thick] (A) arc[start angle=180, end angle=0, radius=2.5cm];
		
		% Draw the diameter
		\draw[thick] (A) -- (B);
		
		% Draw right-angled triangle
		\draw[thick] (A) -- (C) -- (B) -- cycle;

		% Mark the points A, B, C
		\fill[black] (A) circle (2pt) node[below left] {$A$};
		\fill[black] (B) circle (2pt) node[below right] {$B$};
		\fill[black] (C) circle (2pt) node[above right] {$C$};

		% Draw angle arcs
		\draw[thick] ($(A) + (0.5,0)$) arc[start angle=0, end angle=25.0, radius=0.5cm];
		\draw[thick] ($(B) + (-0.5,0)$) arc[start angle=180, end angle=115.0, radius=0.5cm];

		% Label the angles
		\node [above right= 0.3cm and 1cm of A] [anchor=north east] {$\alpha$};
%				\node at ($(A)!0.3!(C)$) [anchor=north east] {$\alpha$};
%		\node at ($(B)!0.6!(C)$) [anchor=north west] {$\beta$};
%		\node at ($(A)!0.5!(B)$) [below] {$\gamma = 90^\circ$};

		\node at ($(A)!0.5!(C)$) [anchor=south east] {$Ankathete$};
		\node at ($(B)!0.5!(C)$) [anchor=west] {$Gegenkathete$};
		\node at ($(A)!0.5!(B)$) [below] {$Hypotenuse$};
		
	\end{tikzpicture}

	\subsection{Sinus und Cosinus}

	
	% Two emphboxes side by side using parbox
	\parbox[t]{0.50\textwidth}{
		\begin{emphbox}
			\begin{align*}
				\sin \alpha &= \frac{GK}{HT} \\
				\sin (-\alpha) &= -\sin \alpha \\
				\sin (90 \degree + \alpha) &= \sin (90 \degree - \alpha)
			\end{align*}
		\end{emphbox}
	}
	\hfill
	\parbox[t]{0.50\textwidth}{
		\begin{emphbox}
			\begin{align*}
				\cos \alpha &= \frac{AK}{HT}\\
				\cos (-\alpha) &= \cos \alpha\\
				\cos (90 \degree + \alpha) &= -\cos (90 \degree - \alpha)
			\end{align*}
		\end{emphbox}
	}
	
	ungerade Funktion
	
	
	
	
	
	\subsection{Sinus}
		\begin{emphbox}
			\begin{align*}
			 \sin \alpha &= \frac{Gegenkathete}{Hypotenuse} \\
			 \sin (-\alpha) &= -\sin \alpha \\
			 \sin (90 \degree + \alpha) &= \sin (90 \degree - \alpha)
			\end{align*}
		\end{emphbox}
		ungerade Funktion
	\subsection{Cosinus}
		\begin{emphbox}
			\begin{align*}
			\cos \alpha &= \frac{Ankathete}{Hypotenuse}\\
			\cos (-\alpha) &= \cos \alpha\\
			\cos (90 \degree + \alpha) &= -\cos (90 \degree - \alpha)
			\end{align*}
		\end{emphbox}
		Gerade Funktion

	\subsection{Tangens}
		\begin{emphbox}
			\begin{align*}
			\tan \alpha &= \frac{Gegenkathete}{Ankathete} \\
			\tan \alpha &= \frac{\sin \alpha}{\cos \alpha} 
			\end{align*}
		\end{emphbox}
		
	\subsection{Cotangens}
		\begin{emphbox}
			\begin{align*}
			\cot \alpha &= \frac{Ankathete}{Gegenkathete} \\
			\cot \alpha &= \frac{\cos \alpha}{\sin \alpha} 		
			\end{align*}
		\end{emphbox}
	mit:\\
	a = Gegenkathete\\
	b = Ankathete\\
	c = Hypotenuse

	\begin{tikzpicture}
		\draw (0, 0) -- (3, 0) -- (2, 2) -- (0, 0);
		\coordinate[label=left:$A$] (A) at (0, 0);
		\coordinate[label=right:$B$] (B) at (3, 0);
		\coordinate[label=above:$C$] (C) at (2, 2);
	\end{tikzpicture}

	\subsection{Trigonometrischer Pythagoras}
		\begin{emphbox}
			$ \sin ^2 \alpha + \cos ^2 \alpha = 1 $
		\end{emphbox}
		
	\subsection{Sinussatz}
		\begin{emphbox}
			$ \frac{a}{\sin \alpha} = \frac{b}{\sin \beta} = \frac{c}{\sin \gamma} = \frac{a \cdot b \cdot c}{2 \cdot A} = 2 \cdot R$
		\end{emphbox}

	\subsection{Cosinussatz}
		\begin{emphbox}
			$ c^2 = a^2 + b^2 - 2 \cdot a \cdot b \cdot \cos \gamma $
		\end{emphbox}

\begin{symbolbox}
	A = Fläche\\
	R = Radius des Umkreises
\end{symbolbox}

\begin{bluebox}
	Test
\end{bluebox}


\end{sectionbox}


\begin{sectionbox}
	\subsection{Cosinus}

	Text goes here ...


\end{sectionbox}


