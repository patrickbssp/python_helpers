\section{Elektrotechnik}

\subsection{Widerstandfarbcodes}
	3 oder 4 Ringe: Z-E-M-T \\
	5 oder 6 Ringe: H-Z-E-M-T-TK \\
	\#: H/Hunderter (100x), Z/Zehner (10x), E/Einer (1x) \\
	M: Multiplikator, T: Toleranz, TK: Temperaturkoeffizient
	\begin{tablebox*}{lllll}
		Farbe & \# & Mult. & Tol. & TK \\
		\hline
		ohne    &   &                        & $\pm 20\%$   & \\
		silber  &   & $10^{-2}$ = 0.01       & $\pm 10\%$   & \\
		gold    &   & $10^{-1}$ = 0.1        & $\pm 5\%$    & \\
		schwarz & 0 & $10^0$ = 1             &              & $200 \cdot 10^{-6}/K$ \\			
		braun   & 1 & $10^1$ = 10            & $\pm 1\%$    & $100 \cdot 10^{-6}/K$ \\
		rot     & 2 & $10^2$ = 100           & $\pm 2\%$    & $50 \cdot 10^{-6}/K$\\
		orange  & 3 & $10^3$ = 1000          &              & $15 \cdot 10^{-6}/K$ \\
		gelb    & 4 & $10^4$ = 10'000        &              & $25 \cdot 10^{-6}/K$ \\
		grün    & 5 & $10^5$ = 100'000       & $\pm 0.5\%$  & \\
		blau    & 6 & $10^6$ = 1'000'000     & $\pm 0.25\%$ & $10 \cdot 10^{-6}/K$ \\
		violett & 7 & $10^7$ = 10'000'000    & $\pm 0.1\%$  & $5 \cdot 10^{-6}/K$ \\
		grau    & 8 & $10^8$ = 100'000'000   & $\pm 0.05\%$ & \\
		weiss   & 9 & $10^9$ = 1'000'000'000 &              & \\
	\end{tablebox*}

\subsection{E-Reihen}
	\begin{tablebox*}{llllllll}
			\multicolumn{8}{l}{E3 (40\% Toleranz)}  \\
			\hline
			1.0 & 2.2 & 4.7 & & & & & \\
			\multicolumn{8}{l}{E6 (20\% Toleranz)}  \\
			\hline
			1.0 & 1.5 & 2.2 & 3.3 & 4.7 & 6.8 \\
			\multicolumn{8}{l}{E12 (10\% Toleranz)}  \\
			\hline
			1.0 & 1.2 & 1.5 & 1.8 & 2.2 & 2.7 \\
			3.3 & 3.9 & 4.7 & 5.6 & 6.8 & 8.2 \\
			\multicolumn{8}{l}{E24 (5\% Toleranz)}  \\
			\hline
			1.0 & 1.1 & 1.2 & 1.3 & 1.5 & 1.6 & 1.8 & 2.0 \\
			2.2 & 2.4 & 2.7 & 3.0 & 3.3 & 3.6 & 3.9 & 4.3 \\
			4.7 & 5.1 & 5.6 & 6.2 & 6.8 & 7.5 & 8.2 & 9.1 \\
			\multicolumn{8}{l}{E48 (2\% Toleranz)}  \\
			\hline
			1.00 & 1.05 & 1.10 & 1.15 & 1.21 & 1.27 & 1.33 & 1.40 \\
			1.47 & 1.54 & 1.62 & 1.69 & 1.78 & 1.87 & 1.96 & 2.05 \\
			2.15 & 2.26 & 2.37 & 2.49 & 2.61 & 2.74 & 2.87 & 3.01 \\
			3.16 & 3.32 & 3.48 & 3.65 & 3.83 & 4.02 & 4.22 & 4.42 \\
			4.64 & 4.87 & 5.11 & 5.36 & 5.62 & 5.90 & 6.19 & 6.49 \\
			6.81 & 7.15 & 7.50 & 7.87 & 8.25 & 8.66 & 9.09 & 9.53 \\
	\end{tablebox*}
	\begin{tablebox*}{llllllll}
			\multicolumn{8}{l}{E96 (1\% Toleranz)} \\
			\hline
			1.00 & 1.02 & 1.05 & 1.07 & 1.10 & 1.13 & 1.15 & 1.18 \\
			1.21 & 1.24 & 1.27 & 1.30 & 1.33 & 1.37 & 1.40 & 1.43 \\
			1.47 & 1.50 & 1.54 & 1.58 & 1.62 & 1.65 & 1.69 & 1.74 \\
			1.78 & 1.82 & 1.87 & 1.91 & 1.96 & 2.00 & 2.05 & 2.10 \\
			2.15 & 2.21 & 2.26 & 2.32 & 2.37 & 2.43 & 2.49 & 2.55 \\
			2.61 & 2.67 & 2.74 & 2.80 & 2.87 & 2.94 & 3.01 & 3.09 \\
			3.16 & 3.24 & 3.32 & 3.40 & 3.48 & 3.57 & 3.65 & 3.74 \\
			3.83 & 3.92 & 4.02 & 4.12 & 4.22 & 4.32 & 4.42 & 4.53 \\
			4.64 & 4.75 & 4.87 & 4.99 & 5.11 & 5.23 & 5.36 & 5.49 \\
			5.62 & 5.76 & 5.90 & 6.04 & 6.19 & 6.34 & 6.49 & 6.65 \\
			6.81 & 6.98 & 7.15 & 7.32 & 7.50 & 7.68 & 7.87 & 8.06 \\
			8.25 & 8.45 & 8.66 & 8.87 & 9.09 & 9.31 & 9.53 & 9.76 \\
			\multicolumn{8}{l}{E192 (0.5\% Toleranz und tiefer)} \\
			\hline
			1.00 & 1.01 & 1.02 & 1.04 & 1.05 & 1.06 & 1.07 & 1.09 \\
			1.10 & 1.11 & 1.13 & 1.14 & 1.15 & 1.17 & 1.18 & 1.20 \\
			1.21 & 1.23 & 1.24 & 1.26 & 1.27 & 1.29 & 1.30 & 1.32 \\
			1.33 & 1.35 & 1.37 & 1.38 & 1.40 & 1.42 & 1.43 & 1.45 \\
			1.47 & 1.49 & 1.50 & 1.52 & 1.54 & 1.56 & 1.58 & 1.60 \\
			1.62 & 1.64 & 1.65 & 1.67 & 1.69 & 1.72 & 1.74 & 1.76 \\
			1.78 & 1.80 & 1.82 & 1.84 & 1.87 & 1.89 & 1.91 & 1.93 \\
			1.96 & 1.98 & 2.00 & 2.03 & 2.05 & 2.08 & 2.10 & 2.13 \\
			2.15 & 2.18 & 2.21 & 2.23 & 2.26 & 2.29 & 2.32 & 2.34 \\
			2.37 & 2.40 & 2.43 & 2.46 & 2.49 & 2.52 & 2.55 & 2.58 \\
			2.61 & 2.64 & 2.67 & 2.71 & 2.74 & 2.77 & 2.80 & 2.84 \\
			2.87 & 2.91 & 2.94 & 2.98 & 3.01 & 3.05 & 3.09 & 3.12 \\
			3.16 & 3.20 & 3.24 & 3.28 & 3.32 & 3.36 & 3.40 & 3.44 \\
			3.48 & 3.52 & 3.57 & 3.61 & 3.65 & 3.70 & 3.74 & 3.79 \\
			3.83 & 3.88 & 3.92 & 3.97 & 4.02 & 4.07 & 4.12 & 4.17 \\
			4.22 & 4.27 & 4.32 & 4.37 & 4.42 & 4.48 & 4.53 & 4.59 \\
			4.64 & 4.70 & 4.75 & 4.81 & 4.87 & 4.93 & 4.99 & 5.05 \\
			5.11 & 5.17 & 5.23 & 5.30 & 5.36 & 5.42 & 5.49 & 5.56 \\
			5.62 & 5.69 & 5.76 & 5.83 & 5.90 & 5.97 & 6.04 & 6.12 \\
			6.19 & 6.26 & 6.34 & 6.42 & 6.49 & 6.57 & 6.65 & 6.73 \\
			6.81 & 6.90 & 6.98 & 7.06 & 7.15 & 7.23 & 7.32 & 7.41 \\
			7.50 & 7.59 & 7.68 & 7.77 & 7.87 & 7.96 & 8.06 & 8.16 \\
			8.25 & 8.35 & 8.45 & 8.56 & 8.66 & 8.76 & 8.87 & 8.98 \\
			9.09 & 9.20 & 9.31 & 9.42 & 9.53 & 9.65 & 9.76 & 9.88
	\end{tablebox*}
	
\begin{sectionbox}
	\subsection{Ohmsches Gesetz}

	\begin{emphbox}
	Ohmsches Gesetz: $ U = R \cdot I = \frac{I}{G} $
	\end{emphbox}

	\subsection{Leistung}

	\begin{emphbox}
	$ P = U \cdot I = \frac{U^2}{R} = I^2 \cdot R $
	\end{emphbox}
	

		
\end{sectionbox}

\begin{sectionbox}
	\subsection{Kirchoffsche Gesetze}

	Knotenregel:
	\begin{emphbox}
		In einem Knoten: Summe der zufliessenden Ströme = Summe der abfliessenden Ströme\\
		$\sum _{k=1}^{n}I_k = 0$
	\end{emphbox}
	
	Maschenregel:
	\begin{emphbox}
		In einer Masche: Summe der Teilspannungen addieren sich zu Null. Pfeilrichtungen beachten!\\
		$\sum _{k=1}^{n}U_k = 0$
	\end{emphbox}

\end{sectionbox}

\begin{sectionbox}
	\subsection{Strom}
	Elektronen bewegen sich entgegengesetzt zur Stromrichtung

	\begin{tabular}{cc}
	\parbox{1cm}{
		\begin{emphbox}
			\begin{align*}
				b &= \frac{v}{E} \\
				J &= \frac{I}{A}
			\end{align*}
		\end{emphbox}
	} & 
	
	\parbox{5cm}{
		\begin{symbolbox}
			$b$: Ladungsträgerbeweglichkeit (in $m^2/Vs$) \\
			$v$: Driftgeschwindigkeit (in m/s) \\
			$E$: el. Feldstärke (in V/m) \\
			$J$: Stromdichte \\
			$I$: Stromstärke \\
			$A$: Leiterquerschnitt
		\end{symbolbox}
	}
\end{tabular}

\end{sectionbox}

\begin{sectionbox}
	\subsection{Widerstand}

	\begin{tabular}{cc}
	\parbox{2.5cm}{
		\begin{emphbox}
			\begin{align*}
				R &= \frac{\rho \cdot l}{A} = \frac{l}{\gamma \cdot A} \\
				\gamma &= \frac{1}{\rho} \\
				R &= \frac{1}{G}
			\end{align*}
		\end{emphbox}
	} & 

	\parbox{3cm}{
		\begin{symbolbox}
			R: Widerstand (in \ohm) \\
			G: Leitwert (in S) \\
			$\rho$ : spez. Widerstand \\
			$\gamma$ : Leitfähigkeit \\
			A : Leiterquerschnitt \\
			l: Leiterlänge
		\end{symbolbox}
	}
	\end{tabular}

	\begin{emphbox}
	
		\begin{tabular}{l|l}
			Parallelschaltung & Serieschaltung \\
			%\hline
			$ \frac{1}{R_{tot}} = \sum\limits _{k=1}^{n}\frac{1}{R_k} $	&
			$ R_{tot} = \sum\limits _{k=1}^{n}R_k $ \\
			$ G_{tot} = \sum\limits _{k=1}^{n}G_k $ & \\
		\end{tabular}
	\end{emphbox}
	
\end{sectionbox}


\begin{sectionbox}
	\subsection{Kondensator}

	Lade-/Entladevorgang:
	\begin{emphbox}
		\begin{align*}
			u_C(t) &= U_0+ ΔU \cdot e^{-{\frac{t}{τ}}} = U_0+ (U_{C,t0}-U_0) \cdot e^{-{\frac{t}{τ}}} \\
			i_C(t) &= \frac{u_C(t)}{R_C} = \frac{U_0}{R_C}+\frac{ΔU}{R_C}\cdot e^{-{\frac{t}{τ}}}
		\end{align*}
	\end{emphbox}




\begin{emphbox}
	\begin{tabular}{l|l}
		Ladevorgang & Entladevorgang \\
		
		$ u_C(t) = U_0 \cdot (1-e^{-{\frac{t}{τ}}}) $ &
		$ u_C(t) = U_0 \cdot e^{-{\frac{t}{τ}}} $ \\

		$ i_C(t) = I_0 \cdot e^{-{\frac{t}{τ}}} $ &
		$ i_C(t) = -I_0 \cdot e^{-{\frac{t}{τ}}} $ \\
		
	\end{tabular}



\end{emphbox}

mit
\begin{emphbox}
	$I_0 = \frac{U_0}{R_C}$
	
	$τ = R_C \cdot C$
\end{emphbox}

	
\end{sectionbox}

\begin{sectionbox}
	\subsection{Spule}

	Lade-/Entladevorgang:
	\begin{emphbox}
		$u_C(t) = U_0+ ΔU \cdot e^{-{\frac{t}{τ}}} = U_0+ (U_{C,t0}-U_0) \cdot e^{-{\frac{t}{τ}}} ???$

		$i_C(t) = \frac{u_C(t)}{R_C} = \frac{U_0}{R_C}+\frac{ΔU}{R_C}\cdot e^{-{\frac{t}{τ}}} ???$
	\end{emphbox}

	\begin{emphbox}
		\begin{tabular}{l|l}
			Ladevorgang & Entladevorgang \\
			$u_L(t) = U_0 \cdot e^{-{\frac{t}{τ}}}$ &
			$u_L(t) = -U_0 \cdot e^{-{\frac{t}{τ}}}$ \\
			$i_L(t) = I_0 \cdot (1-e^{-{\frac{t}{τ}}})$ &
			$i_L(t) = I_0 \cdot e^{-{\frac{t}{τ}}}$ \\
		\end{tabular}
	\end{emphbox}

	mit: $I_0 = \frac{U_0}{R_L}, τ = \frac{L}{R_L}$	

	
\end{sectionbox}


\begin{sectionbox}
	\subsection{Phasenwinkel}

	\begin{emphbox}
		Kreisfrequenz: $\omega = 2\pi f = 2\pi/T$\\
		Phasenwinkel: $\varphi(t) = \omega t + \varphi_0$\\
		Kosinusform: $x(t) = \hat{x} \cos(\omega t + \varphi_0)$\\
	\end{emphbox}

		mit: f = Frequenz\\
		T = Periodendauer\\
		$\varphi_0$ = Nullphasenwinkel, Phasenwinkel zum Zeitpunkt t=0

\end{sectionbox}
	

\begin{sectionbox}
	\subsection{Phasor/komplexe Amplitude}
	Für harmonisch schwingende zeitabhängige Grössen:
	\begin{emphbox}
		$a(t) = \hat{a} \cos(\omega t + \varphi_0)$\\
	
	\end{emphbox}
\end{sectionbox}

\begin{sectionbox}
	\subsection{Frequenzgang}
	\begin{emphbox}
		Verstärkung/Dämpfung: $v = 20\log \frac{G}{G_0}$
	\end{emphbox}
	mit:
	$G$ : Grösse\\
	$G_0$ : Bezugsgrösse\\
	$[v]$ = dB
	
	Grenzfrequenz $f_g$ / Grenzkreisfrequenz $\omega_g$):
	Bei dieser Frequenz ist der Betrag einer frequenzabhängigen Grösse	um $\sqrt{2}$ grösser oder kleiner als eine konstante Bezugsgrösse.
	
	
	Dekade: lin.: $\cdot 10$ o. $/10$, log.: $\pm 20 \si{\dB}$\\
	Oktave:	lin.: $\cdot 2$ o. $/2$, log.: $\pm 6 \si{\dB}$
	\begin{emphbox}
		Normierte Frequenz: $\Omega = \frac{\omega}{\omega_g}$
	\end{emphbox}

\end{sectionbox}

\begin{sectionbox}
	



\end{sectionbox}