\section{Elektrotechnik}
	
	TODO: Preferred Series
	Farbcodes


\subsection{Widerstandfarbcodes}
	\begin{emphbox}
		\begin{tabular}{ccl|ccl}
			Symbol & Präfix & Faktor & Symbol & Präfix & Faktor \\
			\hline
			
			
			

		\end{tabular}
	\end{emphbox}
	

\subsection{E-Reihen}
	\begin{emphbox}
		\begin{tabular}{cccccc}
			E3 & \multicolumn{5}{l}{(40\% Toleranz)}  \\
			\hline
			1.0 & 2.2 & 4.7 & & & \\
			E6 & \multicolumn{5}{l}{(20\% Toleranz)}  \\
			\hline
			1.0 & 1.5 & 2.2 & 3.3 & 4.7 & 6.8 \\
	
\end{tabular}


	\end{emphbox}


	
\begin{sectionbox}
	\subsection{Ohmsches Gesetz}

	\begin{emphbox}
	Ohmsches Gesestz: $ U = R \cdot I = \frac{I}{G} $
	\end{emphbox}

	\subsection{Leistung}

	\begin{emphbox}
	$ P = U \cdot I = \frac{U^2}{R} = I^2 \cdot R $
	\end{emphbox}
	

		
\end{sectionbox}

\begin{sectionbox}
	\subsection{Kirchoffsche Gesetze}

	Knotenregel:
	\begin{emphbox}
		In einem Knoten: Summe der zufliessenden Ströme = Summe der abfliessenden Ströme\\
		$\sum _{k=1}^{n}I_k = 0$
	\end{emphbox}
	
	Maschenregel:
	\begin{emphbox}
		In einer Masche: Summe der Teilspannungen addieren sich zu Null. Pfeilrichtungen beachten!\\
		$\sum _{k=1}^{n}U_k = 0$
	\end{emphbox}

\end{sectionbox}


\begin{sectionbox}
	\subsection{Widerstand}

	\begin{emphbox}
	Leitwert:	$ G = \frac{1}{R} $\\
	\end{emphbox}	
	
	Parallelschaltung:
	\begin{emphbox}
		\begin{align*}
			\frac{1}{R_{tot}} &= \sum _{k=1}^{n}\frac{1}{R_k} &= \frac{1}{R_{1}} + \frac{1}{R_{2}} + \ldots + \frac{1}{R_k}\\
			G_{tot} &= \sum _{k=1}^{n}G_k &= G_1 + G_2 + \ldots + G_k
		\end{align*}
	\end{emphbox}
	
	Serieschaltung:
	\begin{emphbox}
			\begin{align*}
			R_{tot} &= \sum _{k=1}^{n}R_k &= R_1 + R_2 + \ldots + R_k
		\end{align*}

	\end{emphbox}

	
\end{sectionbox}


\begin{sectionbox}
	\subsection{Kondensator}

	Lade-/Entladevorgang:
	\begin{emphbox}
		\begin{align*}
			u_C(t) &= U_0+ ΔU \cdot e^{-{\frac{t}{τ}}} = U_0+ (U_{C,t0}-U_0) \cdot e^{-{\frac{t}{τ}}} \\
			i_C(t) &= \frac{u_C(t)}{R_C} = \frac{U_0}{R_C}+\frac{ΔU}{R_C}\cdot e^{-{\frac{t}{τ}}}
		\end{align*}
	\end{emphbox}

	Ladevorgang:
	\begin{emphbox}
		\begin{align*}
			u_C(t) &= U_0 \cdot (1-e^{-{\frac{t}{τ}}}) \\
			i_C(t) &= I_0 \cdot e^{-{\frac{t}{τ}}}
		\end{align*}
	\end{emphbox}

	Entladevorgang:
	\begin{emphbox}
		\begin{align*}
			u_C(t) &= U_0 \cdot e^{-{\frac{t}{τ}}} \\
			i_C(t) &= -I_0 \cdot e^{-{\frac{t}{τ}}}
		\end{align*}
	\end{emphbox}

mit
\begin{emphbox}
	$I_0 = \frac{U_0}{R_C}$
	
	$τ = R_C \cdot C$
\end{emphbox}

	
\end{sectionbox}

\begin{sectionbox}
	\subsection{Spule}

Lade-/Entladevorgang:
\begin{emphbox}
	$u_C(t) = U_0+ ΔU \cdot e^{-{\frac{t}{τ}}} = U_0+ (U_{C,t0}-U_0) \cdot e^{-{\frac{t}{τ}}} ???$

	$i_C(t) = \frac{u_C(t)}{R_C} = \frac{U_0}{R_C}+\frac{ΔU}{R_C}\cdot e^{-{\frac{t}{τ}}} ???$
\end{emphbox}

Ladevorgang:
\begin{emphbox}

	$i_L(t) = I_0 \cdot (1-e^{-{\frac{t}{τ}}})$

	$u_L(t) = U_0 \cdot e^{-{\frac{t}{τ}}}$

\end{emphbox}

Entladevorgang:
\begin{emphbox}
	$u_L(t) = -U_0 \cdot e^{-{\frac{t}{τ}}}$

	$i_L(t) = I_0 \cdot e^{-{\frac{t}{τ}}}$
\end{emphbox}

mit
\begin{emphbox}
	$I_0 = \frac{U_0}{R_L}$
	
	$τ = \frac{L}{R_L}$	
\end{emphbox}

	
\end{sectionbox}