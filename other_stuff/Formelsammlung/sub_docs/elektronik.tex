\section{Elektronik}

\subsection{Operationsverstärker}

\subsubsection{Allgemein}
	
	\resizebox{3cm}{!}{	
	\begin{circuitikz}[european]
		\draw (0, 5) node[op amp,yscale=-1] (opamp) {}
		(opamp.-) to[short,-o] ++(-0.5, 0) coordinate(N)
		(opamp.+) to[short,-o] ++(-1, 0) coordinate(P)
		(opamp.out) to[short,-o] ++(0.5, 0) coordinate(O)
		(O |- 0,4) node[rground](OG){} node[ocirc] {}
		(OG -| N) node[rground](ON){} node[ocirc]{}
		(OG -| P) node[rground](OP){} node[ocirc]{}
		% draw voltage arrows
		(P) to [open, v_=$u_+$, voltage=straight] (OP)
		(N) to [open, v^=$u_-$, voltage=straight] (ON)
		(O) to [open, v^=$u_{out}$, voltage=straight] (OG)
		;
	\end{circuitikz}
	} % resizebox

	\begin{tabular}{cc}
	\parbox{2.5cm}{
		\begin{emphbox}
			\begin{align*}
				u_{out} &= A_{OL} \cdot (u_+ - u_-)
			\end{align*}
		\end{emphbox}
	} & 
	
	\parbox{3cm}{
		\begin{symbolbox}
			$A_{OL}$: Open-loop gain
		\end{symbolbox}
	}
\end{tabular}

\subsubsection{Nichtinvertierender Verstärker (Elektrometerverstärker)}

	\resizebox{3cm}{!}{	
			\begin{circuitikz}[european]
	\draw (0, 5) node[op amp,yscale=-1] (op) {}
	(op.+) to[short,-o] ++(-1, 0) coordinate(P)
	(op.out) to[short,-o] ++(1, 0) coordinate(O)
	(op.out) node[circ] {} to[european resistor, l=$R_2$] ++(0, -2) coordinate(R)
	to[european resistor, l=$R_1$] ++(0, -2)
	node[rground](G){}
	(O |- G) node[rground](OG){} node[ocirc] {}
	(R) to[short,*-] (op.- |- R) to[short] (op.-)
	%		(OG -| N) node[rground](ON){} node[ocirc]{}
	(OG -| P) node[rground](OP){} node[ocirc]{}
	% draw voltage arrows
	(P) to [open, v_=$u_{in}$, voltage=straight] (OP)
	(O) to [open, v^=$u_{out}$, voltage=straight] (OG)
	;
\end{circuitikz}
	} % resizebox

	\begin{emphbox}
		\begin{align*}
			U_{out} &= (1 + \frac{R_2}{R_1}) \cdot U_{in}
		\end{align*}
	\end{emphbox}

\subsubsection{Spannungsfolger}

Spezialfall des nichtinv. Verstärkers mit $R_2 = 0\ohm$ und $R_1 = \infty\ohm$.


\subsubsection{Invertierender Addierer/Summierverstärker}

	\resizebox{6cm}{!}{
		\begin{tikzpicture}
	% Paths, nodes and wires:
	\coordinate (I3) at (0,7);
	\coordinate (I2) at (-1,8);
	\coordinate (I1) at (-2,9);
	\coordinate (GO) at (7,5);
	\draw node[op amp](op) at (4.2, 6.5) {};
	\draw (I1 -| op.+) to[european resistor, l=$R_f$] (I1 -| op.out) to[short,-*] (op.out);
	\draw (I1) to[short,o-] (I1 -| I3) to[european resistor, l=$R_1$] (I1 -| op.-);
	\draw (I2) to[short,o-] (I2 -| I3) to[european resistor, l=$R_2$] (I2 -| op.-);
	\draw (I3) to[european resistor, l=$R_3$] (I3 -| op.-);
	\draw (3, 8) to[short,-*] (3, 9);
	\draw (3, 7) to[short,*-*] (3, 8);
	
	\draw (I3) node[ocirc] {};
	
	\draw (op.out) to[short,-o] (op.out -| GO) to[open,v^=$U_{out}$, voltage=straight] (GO) node[rground] {};
	
	\draw (5.4, 6.5) -| (6, 6.5);
	\draw (op.+) -- (op.+ |- GO) node[rground]{};
	
	% draw voltage arrows and ground symbols, aligned to GO
	\draw (I3) to [open, v^=$U_3$, voltage=straight] (GO -| I3)
	node[rground] {} node[ocirc] {};
	\draw (I2) to [open, v^=$U_2$, voltage=straight] (GO -| I2)
	node[rground] {} node[ocirc] {};
	\draw (I1) to [open, v^=$U_1$, voltage=straight] (GO -| I1)
	node[rground] {} node[ocirc] {};
\end{tikzpicture}
	}
	
	\begin{emphbox}
		\begin{align*}
			U_{out} &= -R_f \cdot (\frac{U_1}{R_1} + \frac{U_2}{R_2} + \frac{U_3}{R_3})
		\end{align*}
	\end{emphbox}

\subsubsection{Invertierender Verstärker}

Der invertierende Verstärker ist ein Spezialfall des inv. Addierers mit einem einzelnen Eingang.

\subsubsection{Differenzverstärker/Subtrahierverstärker}

	\resizebox{6cm}{!}{
		\begin{circuitikz}
	% Paths, nodes and wires:
	
	\draw node[op amp](op) at (4.2, 6.5) {};
	\draw (op.-) to[short,*-] ++(0,1) coordinate(N) to[european resistor,l=$R_f$] (N-|op.out) to[short,-*] (op.out);
	
	\draw (op.+) node[circ]{} to[european resistor,l=$R_g$] ++(0, -2) coordinate(G) node[rground] {};
	
	\draw (op.+) to[european resistor, l=$R_2$] ++(-2, 0) coordinate(I2) node[ocirc]{} to[open,v^=$U_2$, voltage=straight] (G-|I2) node[ocirc]{} node[rground] {};
	
	\draw (op.-) to[european resistor, l=$R_1$]  ++(-2, 0) to[short] ++(-0.5, 0) coordinate(I1) node[ocirc]{} to[open,v^=$U_1$, voltage=straight] (G-|I1) node[ocirc]{} node[rground] {};
	
	\draw (op.out) to[short,-o] ++(0.5,0) coordinate(O) to[open,v^=$U_{out}$, voltage=straight] (O|-G) node[ocirc]{} node[rground] {};
	
\end{circuitikz}
	}

	\begin{emphbox}
		\begin{align*}
			not checked!
%			U_{out} &= U_2 \cdot \frac{R_1+R_f}{R_1} \cdot \frac{R_g}{R_2+R_g} - U_1 \frac{R_f}{R_1}
		\end{align*}
	\end{emphbox}

\subsubsection{Instrumentenverstärker}
	\resizebox{6cm}{!}{
		\begin{circuitikz}
	% Paths, nodes and wires:
	
	\draw node[op amp,yscale=-1](op1) at (1.2, 6.5) {};
	\draw node[op amp](op2) at (1.2, 0.5) {};
	\draw node[op amp](op3) at (5.6, 3.5) {};
	\draw (op1.out) to[european resistor, l=$R_1$] ++(0,-2) coordinate (N1) -| (op1.-);
	\draw (N1) node[circ]{} to[european resistor, l=$R_g$]  ++(0,-2) coordinate (N2) node[circ]{};
	\draw (op2.out) to[european resistor, l=$R_1$] ++(0,2) (N2) -| (op2.-);
	\draw (op1.out) to[european resistor, l=$R_2$, *-] ++(2,0) coordinate (N3) -- (op3.-);
	\draw (N3) to[european resistor, l=$R_3$, *-] ++(2,0) -| (op3.out) to[short,*-] ++(0.5,0) node[ocirc]{} to[open,v^=$U_{out}$, voltage=straight] ++(0,-5) coordinate(GND) node[ocirc]{} node[rground] {};
	\draw (op2.out) to[european resistor, l=$R_2$, *-] ++(2,0) coordinate (N4) -- (op3.+);
	
	\draw (op1.+) -- ++(-0.5,0) coordinate(I1) node[ocirc]{} to[open,v^=$U_1$, voltage=straight] ++(0,-2) node[ocirc]{} node[rground] {};
	
	\draw (op2.+) -- ++(-0.5,0) coordinate(I2) node[ocirc]{} to[open,v^=$U_2$, voltage=straight] ++(0,-2) node[ocirc]{} node[rground] {};
	
	\draw (N4) to[european resistor, l=$R_3$, *-] ++(2,0) coordinate(N5) -- (N5 |- GND) node[rground]{};
\end{circuitikz}
	}
	\begin{emphbox}
		\begin{align*}
			U_{out} &= (1+\frac{2R}{Rg}) \cdot \frac{R_3}{R_2} \cdot (U_2-U_1)
		\end{align*}
	\end{emphbox}
\subsubsection{Integrierer}
\subsubsection{Differenzierer}
\subsubsection{Tiefpass 1. Ordnung}

	\begin{circuitikz}
		% Paths, nodes and wires:
		\coordinate (I) at (0, 1);
		\draw (I) to[european resistor, l=$R$, o-*] ++(2,0) coordinate(N) -- ++(1,0) coordinate(O) node[ocirc]{};
		\draw (N) to[capacitor] ++(0,-2) coordinate(N2) node[ocirc]{};
		\draw (I|-N2) to[short,o-o] (O|-N2);
		\draw (I) to[open,v^=$U_1$, voltage=straight] (I|-N2);
		\draw (O) to[open,v^=$U_2$, voltage=straight] (O|-N2);
	\end{circuitikz}
