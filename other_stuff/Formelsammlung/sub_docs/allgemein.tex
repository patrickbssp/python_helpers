\section{Einheiten}

\begin{sectionbox}
	\subsection{SI-Präfixe}
		\begin{emphbox}
			\begin{tabular}{ccl|ccl}
				Symbol & Präfix & Faktor & Symbol & Präfix & Faktor \\

				Q  & Quetta & $10^{30}$ & d & Dezi & $10^{-1}$\\
				R  & Ronna  & $10^{27}$ & c & Zenti & $10^{-2}$\\
				Y  & Yotta  & $10^{24}$ & m & Milli & $10^{-3}$\\
				Z  & Zetta  & $10^{21}$ & µ & Mikro & $10^{-6}$\\ 
				E  & Exa    & $10^{18}$ & n & Nano  & $10^{-9}$\\
				P  & Peta   & $10^{15}$ & p & Piko  & $10^{-12}$\\
				T  & Tera   & $10^{12}$ & f & Femto  & $10^{-15}$\\
			 	G  & Giga   & $10^9$    & a & Atto   & $10^{-18}$\\
				M  & Mega   & $10^6$    & z & Zepto  & $10^{-21}$\\
				k  & Kilo   & $10^3$    & y & Yokto  & $10^{-24}$\\
				h  & Hekto  & $10^2$    & r & Ronto  & $10^{-27}$\\
				da & Deka   & $10^1$    & q & Quekto & $10^{-30}$\\
			\end{tabular}
		\end{emphbox}

	\subsection{Binäre Präfixe}
		\begin{emphbox}
			\begin{tabular}{ccll}
				Symbol & Präfix & Faktor & Dez. Äquiv. \\
			
				Ki & Kibi  & $2^{10}$  & $= 1.024e3$ \\
				Mi & Mebi  & $2^{20}$  & $≈ 1.049e6$ \\
				Gi & Gibi  & $2^{30}$  & $≈ 1.074e9$ \\
				Ti & Tebi  & $2^{40}$  & $≈ 1.100e12$ \\
				Pi & Pebi  & $2^{50}$  & $≈ 1.126e15$ \\
				Ei & Exbi  & $2^{60}$  & $≈ 1.152e18$ \\
				Zi & Zebi  & $2^{70}$  & $≈ 1.181e21$ \\
				Yi & Yobi  & $2^{80}$  & $≈ 1.209e24$ \\
				Ri & Robi  & $2^{90}$  & $≈ 1.238e27$ \\
				Qi & Quebi & $2^{100}$ & $≈ 1.268e30$ \\

			\end{tabular}
		\end{emphbox}
	
	\subsection{SI-Einheiten}
		\begin{emphbox}
			\begin{tabular}{llc}
				Grösse & Einheit & Symbol \\
				Länge & Meter & m \\
				Masse & Kilogramm & kg \\
				Zeit & Sekunde & s \\
				Stromstärke & Ampere & A \\
				Temperatur & Kelvin & K \\
				Stoffmenge & - & mol \\
				Lichtstärke & Candela & cd			
		\end{tabular}
	\end{emphbox}
	
	
	\subsection{Abgeleitete Einheiten}
		\begin{emphbox}
			\begin{tabular}{lcllc}
				Grösse & Symbol & Einheit & Einheit & SI \\
				Kraft & F & Newton & N & $kg\cdot m/s^2$ \\
				Energie/Arbeit & E/W & Joule & J & $Nm$ \\
				Leistung & P & Watt & W & $J/s$ \\
				El. Ladung & Q & Coulomb & C & $A \cdot s$ \\
				El. Potential & $\phi$ & Volt & V & $J/C$ \\
				Spannung & U & Volt & V & $J/C$ \\
				El. Widerstand & R & Ohm & $\ohm$ & $V/A$ \\
				El. Leitwert & G & Siemens & S & $A/V$ \\
				Spez. el. Widerstand & $\rho$ & - & - & $Ωm$ \\
				Spez. el. Leitwert & $\gamma$ & - & - & $S/m$ \\
				El. Feldstärke & E & - & - & $V/m$ \\
				El. Stromdichte & J & - & - & $A/m^2$ \\
			\end{tabular}
		\end{emphbox}
	
\end{sectionbox}

\begin{sectionbox}
	\subsection{Mach}
	
		\begin{emphbox}
				\begin{align*}
					\sin \mu &= \frac{c}{v} = \frac{1}{m}\\
					\mu &= \arcsin \frac{1}{M}
				\end{align*}
				mit:\\
					c: Schallgeschwindigkeit\\
					v: Objektgeschwindigkeit\\
					$\mu: Machwinkel$
		\end{emphbox}

%		\centering\noindent%

	\resizebox{5cm}{!}{
			
			\def\dotMarkRightAngle[size=#1](#2,#3,#4){%
				\draw ($(#3)!#1!(#2)$) -- 
				($($(#3)!#1!(#2)$)!#1!90:(#2)$) --
				($(#3)!#1!(#4)$);
				\path (#3) --node[circle,fill,inner sep=.5pt]{} ($($(#3)!#1!(#2)$)!#1!90:(#2)$);
			}

			\begin{tikzpicture}
				\coordinate (A) at (0,0);				
				\coordinate (B) at (4,0);
				\coordinate (C) at (0,3);
				\coordinate (D) at ($(C)!(A)!(B)$);
		
				\draw [-Stealth] (A) -- node[below] {Objektgeschwindigkeit v} (B);
				\draw [-Stealth] (A) -- node[below] {Schallgeschwindigkeit c} (D);
				\draw [dashed] (B) -- node[above] {Machwelle} (C);

				\pic [draw, angle radius=5mm, angle eccentricity=1.2mm] {angle = D--B--A};
				
				\node[above left = 0.1cm and 0.8cm of B] {$\mu$};

				\dotMarkRightAngle[size=6pt](A,D,B);			
			\end{tikzpicture}
			
	}		
	\resizebox{5cm}{!}{
	
	\def\dotMarkRightAngle[size=#1](#2,#3,#4){%
		\draw ($(#3)!#1!(#2)$) -- 
		($($(#3)!#1!(#2)$)!#1!90:(#2)$) --
		($(#3)!#1!(#4)$);
		\path (#3) --node[circle,fill,inner sep=.5pt]{} ($($(#3)!#1!(#2)$)!#1!90:(#2)$);
	}
	
	\begin{tikzpicture}
		\coordinate (A) at (0,0);				
		\coordinate (B) at (4,0);
		\coordinate (C) at (0,3);
		\coordinate (D) at ($(C)!(A)!(B)$);
		
		\draw [-Stealth] (A) -- node[below] {Objektgeschwindigkeit v} (B);
		\draw [-Stealth] (A) -- node[below] {Schallgeschwindigkeit c} (D);
		\draw [dashed] (B) -- node[above] {Machwelle} (C);
		
		\pic [draw, angle radius=5mm, angle eccentricity=1.2mm] {angle = D--B--A};
		
		\node[above left = 0.1cm and 0.8cm of B] {$\mu$};
		
		\dotMarkRightAngle[size=6pt](A,D,B);
		
		\draw (2,0) circle (2cm);
		\draw (3,0) circle (1cm);
		
	\end{tikzpicture}
	
}		
	$M < 1.0$: Subsonic \\
	$M = 1.0$: Transonic \\
	$M > 1.0$: Supersonic \\
	$M > 5.0$: Hypersonic
	
\end{sectionbox}