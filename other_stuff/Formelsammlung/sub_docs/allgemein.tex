\section{Einheiten}
%\sectionbox{
%\subsection{SI-Einheiten}
%\symbolbox{

%}
%}
\sectionbox{
\topicbox{SI-Präfixe}{
	\begin{tabular}{ccl|ccl}
	Symbol & Vorsatz & Faktor & Symbol & Vorsatz & Faktor \\
	Y&Yotta&$10^{24}$& d &Dezi&$10^{-1}$\\
	Z&Zetta&$10^{21}$& c &Zenti&$10^{-2}$\\
	E&Exa&$10^{18}$& m & Milli&$10^{-3}$\\
	P&Peta&$10^{15}$& $\mu$ & Mikro&$10^{-6}$\\
	T&Tera&$10^{12}$& n &Nano&$10^{-9}$\\
 	G&Giga&$10^9$& p &Piko&$10^{-12}$\\
	M&Mega&$10^6$& f &Femto&$10^{-15}$\\
	k&Kilo&$10^3$& a &Atto&$10^{-18}$\\
	h&Hekto&$10^2$& z & Zepto&$10^{-21}$\\
	da&Deka&$10^1$ & y &Yokto&$10^{-24}$\\
	\end{tabular}
}
}


\section{Trigonometrie}

\begin{sectionbox}
	\subsection{Sinus}
		\begin{emphbox}
			\begin{align*}
			 \sin \alpha &= \frac{Gegenkathete}{Hypotenuse} \\
			 \sin (-\alpha) &= -\sin \alpha \\
			 \sin (90 \degree + \alpha) &= \sin (90 \degree - \alpha)
			\end{align*}
		\end{emphbox}
		ungerade Funktion
	\subsection{Cosinus}
		\begin{emphbox}
			\begin{align*}
			\cos \alpha &= \frac{Ankathete}{Hypotenuse}\\
			\cos (-\alpha) &= \cos \alpha\\
			\cos (90 \degree + \alpha) &= -\cos (90 \degree - \alpha)
			\end{align*}
		\end{emphbox}
		Gerade Funktion

	\subsection{Tangens}
		\begin{emphbox}
			\begin{align*}
			\tan \alpha &= \frac{Gegenkathete}{Ankathete} \\
			\tan \alpha &= \frac{\sin \alpha}{\cos \alpha} 
			\end{align*}
		\end{emphbox}
		
	\subsection{Cotangens}
		\begin{emphbox}
			\begin{align*}
			\cot \alpha &= \frac{Ankathete}{Gegenkathete} \\
			\cot \alpha &= \frac{\cos \alpha}{\sin \alpha} 		
			\end{align*}
		\end{emphbox}
	mit:\\
	a = Gegenkathete\\
	b = Ankathete\\
	c = Hypotenuse

	\begin{tikzpicture}
		\draw (0, 0) -- (3, 0) -- (2, 2) -- (0, 0);
		\coordinate[label=left:$A$] (A) at (0, 0);
		\coordinate[label=right:$B$] (B) at (3, 0);
		\coordinate[label=above:$C$] (C) at (2, 2);
	\end{tikzpicture}

	\subsection{Trigonometrischer Pythagoras}
		\begin{emphbox}
			$ \sin ^2 \alpha + \cos ^2 \alpha = 1 $
		\end{emphbox}
		
	\subsection{Sinussatz}
		\begin{emphbox}
			$ \frac{a}{\sin \alpha} = \frac{b}{\sin \beta} = \frac{c}{\sin \gamma} = \frac{a \cdot b \cdot c}{2 \cdot A} = 2 \cdot R$
		\end{emphbox}

	\subsection{Cosinussatz}
		\begin{emphbox}
			$ c^2 = a^2 + b^2 - 2 \cdot a \cdot b \cdot \cos \gamma$
		\end{emphbox}

\begin{symbolbox}
	A = Fläche\\
	R = Radius des Umkreises
\end{symbolbox}

\begin{bluebox}
	Test
\end{bluebox}


\end{sectionbox}


\begin{sectionbox}
	\subsection{Cosinus}

	Text goes here ...


\end{sectionbox}


