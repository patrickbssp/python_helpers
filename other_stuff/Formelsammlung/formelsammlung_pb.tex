% % % % % % % % % % % % % % % % % % % % % % % % % % % % % % % % % % % % % % % %
% LaTeX4EI Template for Cheat Sheets                                Version 1.1
%
% Authors: Emanuel Regnath, Martin Zellner
% Contact: info@latex4ei.de
% Encode: UTF-8, tabwidth = 4, newline = LF
% % % % % % % % % % % % % % % % % % % % % % % % % % % % % % % % % % % % % % % %


% ======================================================================
% Document Settings
% ======================================================================

% possible options: color/nocolor, english/german, threecolumn
% defaults: color, english
\documentclass[english]{latex4ei/latex4ei_sheet}
\usepackage{amsmath}
\usepackage{gensymb}
\usepackage{tikz}
\usetikzlibrary{arrows.meta,calc,angles,positioning}
\usepackage{circuitikz}
\ctikzset{voltage/distance from node=.2}% defines arrow's distance from nodes
\ctikzset{voltage/distance from line=.02}% defines arrow's distance from wires
\ctikzset{voltage/bump b/.initial=.1}% defines arrow's curvature
\usepackage{bodegraph}
\usepackage{tabularx}

\def\centerarc[#1](#2)(#3:#4:#5)
% Syntax: [draw options] (center) (initial angle:final angle:radius)
{ \draw[#1] ($(#2)+({#5*cos(#3)},{#5*sin(#3)})$) arc (#3:#4:#5); }

\makeatletter
\newcommand\currentcoordinate{\the\tikz@lastxsaved,\the\tikz@lastysaved}
\makeatother

% set document information
\title{LaTeX4EI \\ Cheat Sheet}
\author{LaTeX4EI}					% optional, delete if unchanged
\myemail{info@latex4ei.de}			% optional, delete if unchanged
\mywebsite{www.latex4ei.de}			% optional, delete if unchanged


% ======================================================================
% Begin
% ======================================================================
\begin{document}

% Title
% ----------------------------------------------------------------------
%\maketitle   % requires ./img/Logo.pdf

% Section
% ----------------------------------------------------------------------

\section{Einheiten}

\begin{sectionbox}
	\subsection{SI-Präfixe}
	
	\begin{tablebox}{ccl|ccl}
		Symbol & Präfix & Faktor & Symbol & Präfix & Faktor \\
		\hline
		Q  & Quetta & $10^{30}$ & d & Dezi & $10^{-1}$\\
		R  & Ronna  & $10^{27}$ & c & Zenti & $10^{-2}$\\
		Y  & Yotta  & $10^{24}$ & m & Milli & $10^{-3}$\\
		Z  & Zetta  & $10^{21}$ & µ & Mikro & $10^{-6}$\\ 
		E  & Exa    & $10^{18}$ & n & Nano  & $10^{-9}$\\
		P  & Peta   & $10^{15}$ & p & Piko  & $10^{-12}$\\
		T  & Tera   & $10^{12}$ & f & Femto  & $10^{-15}$\\
	 	G  & Giga   & $10^9$    & a & Atto   & $10^{-18}$\\
		M  & Mega   & $10^6$    & z & Zepto  & $10^{-21}$\\
		k  & Kilo   & $10^3$    & y & Yokto  & $10^{-24}$\\
		h  & Hekto  & $10^2$    & r & Ronto  & $10^{-27}$\\
		da & Deka   & $10^1$    & q & Quekto & $10^{-30}$\\
	\end{tablebox}

	\subsection{Binäre Präfixe}
	
	\begin{tablebox}{ccll}
		Symbol & Präfix & Faktor & Dez. Äquiv. \\
		\hline			
		Ki & Kibi  & $2^{10}$  & $= 1.024e3$ \\
		Mi & Mebi  & $2^{20}$  & $≈ 1.049e6$ \\
		Gi & Gibi  & $2^{30}$  & $≈ 1.074e9$ \\
		Ti & Tebi  & $2^{40}$  & $≈ 1.100e12$ \\
		Pi & Pebi  & $2^{50}$  & $≈ 1.126e15$ \\
		Ei & Exbi  & $2^{60}$  & $≈ 1.152e18$ \\
		Zi & Zebi  & $2^{70}$  & $≈ 1.181e21$ \\
		Yi & Yobi  & $2^{80}$  & $≈ 1.209e24$ \\
		Ri & Robi  & $2^{90}$  & $≈ 1.238e27$ \\
		Qi & Quebi & $2^{100}$ & $≈ 1.268e30$ \\
	\end{tablebox}

	\subsection{SI-Einheiten}

	\begin{tablebox}{llc}
		Grösse & Einheit & \\
		\hline
		Länge & Meter & m \\
		Masse & Kilogramm & kg \\
		Zeit & Sekunde & s \\
		Stromstärke & Ampere & A \\
		Temperatur & Kelvin & K \\
		Stoffmenge & - & mol \\
		Lichtstärke & Candela & cd			
	\end{tablebox}	
	
	\subsection{Abgeleitete Einheiten}

	\begin{tablebox}{lcllc}
		Grösse & Sym. & Einheit &  & SI \\
		\hline
		Kraft & F & Newton & N & $kg\cdot m/s^2$ \\
		Energie/Arbeit & E/W & Joule & J & $Nm$ \\
		Leistung & P & Watt & W & $J/s$ \\
		El. Ladung & Q & Coulomb & C & $A \cdot s$ \\
		El. Potential & $\phi$ & Volt & V & $J/C$ \\
		Spannung & U & Volt & V & $J/C$ \\
		El. Widerstand & R & Ohm & $\ohm$ & $V/A$ \\
		El. Leitwert & G & Siemens & S & $A/V$ \\
		Spez. el. Widerstand & $\rho$ & - & - & $Ωm$ \\
		Spez. el. Leitwert & $\gamma$ & - & - & $S/m$ \\
		El. Feldstärke & E & - & - & $V/m$ \\
		El. Stromdichte & J & - & - & $A/m^2$ \\
	\end{tablebox}
		
\end{sectionbox}

\section{Mathematik}

\subsection{Vektoren}

Darstellung:
	$\overrightarrow{a} = \mathbf{a} = \begin{bmatrix} a_1, a_2, a_3 \end{bmatrix} = \begin{bmatrix} a_1 \\ a_2 \\ a_3 \end{bmatrix}$

\begin{sectionbox}
	
	O bezeichnet den Ursprung (origo)
	
	\begin{tikzpicture}
		% Draw grid
		\draw[help lines, color=gray!30] (-1,-1) grid (5,4);
		
		% Draw axes
		\draw[thick,->] (-0.5,0) -- (3.5,0) node[right] {$x$};
		\draw[thick,->] (0,-0.5) -- (0,3.5) node[above] {$y$};
		
		% Define points
		\coordinate (P) at (1,1);
		\coordinate (Q) at (2,3);
		
		% Draw vectors
		\draw[thick,->,blue] (0,0) -- (P) node[midway,below right] {$\vec{p}$};
		\draw[thick,->,red] (0,0) -- (Q) node[midway,above left] {$\vec{q}$};
		\draw[thick,->,green] (P) -- (Q) node[midway,right] {$\vec{r}$};
		% Draw and label points
		\fill (P) circle (2pt) node[above right] {$P (1,1)$};
		\fill (Q) circle (2pt) node[above right] {$Q (2,3)$};
		
	\end{tikzpicture}
	
	\begin{emphbox}
		\begin{align*}
			\overrightarrow{p} &= \overrightarrow{OP} &= \overrightarrow{P} \\
			\overrightarrow{q} &= \overrightarrow{OQ} &= \overrightarrow{Q} \\
			\overrightarrow{r} &= \overrightarrow{PQ} &= \overrightarrow{OQ} - 		\overrightarrow{OP} = \overrightarrow{q} - \overrightarrow{p}
		\end{align*}
	\end{emphbox}

	Betrag/Länge eines Vektors
	\begin{emphbox}
	\begin{align*}
		a = |\overrightarrow{a}| = \sqrt{\overrightarrow{a} {\cdot} \overrightarrow{a}} = \sqrt{a_1^2 + a_2^2 + a_3^2} 
	\end{align*}
\end{emphbox}
	
	Rechenregeln
	\begin{emphbox}
		\begin{tabular}{l|l}
			Vektoraddition/-subtraktion & 	Multiplikation mit Skalar \\
			$\overrightarrow{a} \pm \overrightarrow{b} = \begin{bmatrix} a_1 \pm b_1 \\ a_2 \pm b_2 \\ a_3 \pm b_3 \end{bmatrix} $ & $
			c {\cdot} \overrightarrow{a} = \begin{bmatrix} c {\cdot} a_1 \\ c {\cdot} a_2 \\ c {\cdot} a_3 \end{bmatrix} $ \\

			\text{Kommutativgesetz:} & $ \overrightarrow{a} + \overrightarrow{b} = \overrightarrow{b} + \overrightarrow{a} $ \\
			\text{Assoziativgesetz:} & $ (\overrightarrow{a} + \overrightarrow{b}) + \overrightarrow{c} = \overrightarrow{a} + (\overrightarrow{b} + \overrightarrow{c}) $ \\
			\text{Distributivgesetz:} & $ \lambda {\cdot} (\overrightarrow{a} + \overrightarrow{b}) = \lambda {\cdot} \overrightarrow{a} + \lambda {\cdot} \overrightarrow{b} $
		\end{tabular}
	\end{emphbox}

	Einheitsvektor \\
	- markiert mit Zirkumflex \\
	- Zeigt in gleiche Richtung, aber mit Länge 1
	\begin{emphbox}
		\begin{align*}
	 		\hat{a} &= \frac{1}{|\overrightarrow{a}|} {\cdot} \overrightarrow{a} \\
	 		|\hat{a}| &= 1
		\end{align*}
	\end{emphbox}

	
	Normalenvektor

	Steht orthogonal auf einer Geraden, Kurve oder Ebene

	Normaleneinheitsvektor
	
	Normalenvektor mit Länge 1.
	
	\begin{emphbox}
		\begin{align*}
			\overrightarrow{n_0} = \frac{1}{|\overrightarrow{n}|} {\cdot} \overrightarrow{n} \text{Vektor mit Richtung von a, mit Länge 1}
		\end{align*}
	\end{emphbox}
	mit: $\overrightarrow{n_0}$: Normaleneinheitsvektor, $\overrightarrow{n}$: Normalenvektor

	Nullvektor
	Richtung undefiniert, Länge = 0

\end{sectionbox}
\begin{sectionbox}
	Skalarprodukt (Punktprodukt, inneres Produkt)
	\begin{emphbox}
		\begin{tabular}{lcl}
			\text{Geom. Form:} & $ \overrightarrow{a} {\cdot} \overrightarrow{b} $ & $ = \lvert \overrightarrow{a} \rvert {\cdot} |\overrightarrow{b}| {\cdot} \cos \alpha $ \\

			\text{Koord.form:} & $ \overrightarrow{a} {\cdot} \overrightarrow{b} $ &= $ \begin{bmatrix} a_1 \\ a_2 \\ a_3 \end{bmatrix} {\cdot} \begin{bmatrix} b_1 \\ b_2 \\ b_3 \end{bmatrix} = a_1 {\cdot} b_1 + a_2 {\cdot} b_2 + a_3 {\cdot} b_3 $ \\

			\text{...} & $ \overrightarrow{a} \perp \overrightarrow{b}$ & $\Leftrightarrow \overrightarrow{a} {\cdot} \overrightarrow{b} = 0 $ \\

			\text{Kommutativgesetz:} & $ \overrightarrow{a} {\cdot} \overrightarrow{b}$ &= $\overrightarrow{b} {\cdot} \overrightarrow{a}$ \\

			\text{Assoziativgesetz:} & $(\lambda {\cdot} \overrightarrow{a}) {\cdot} \overrightarrow{b}$ &= $\lambda {\cdot} (\overrightarrow{a} {\cdot} \overrightarrow{b})$ \\

			\text{Distributivgesetz:} &	$\overrightarrow{a} {\cdot} (\overrightarrow{b} + \overrightarrow{c})$ &= $\overrightarrow{a} {\cdot} \overrightarrow{b} + \overrightarrow{a} {\cdot} \overrightarrow{c}$
		\end{tabular}
	\end{emphbox}

	Kreuzprodukt (Vektorprodukt, äusseres Produkt)
	\begin{emphbox}
		\begin{tabular}{lcl}
			\text{Geom. Form:} & 
			$\overrightarrow{a} \times \overrightarrow{b}$ &= $|\overrightarrow{a}| {\cdot} |\overrightarrow{b}| {\cdot} \sin \alpha
			{\cdot} \overrightarrow{n}$ \\

			\text{Koord.form:} & $\overrightarrow{a} \times \overrightarrow{b}$ &= $\begin{bmatrix} a_1 \\ a_2 \\ a_3 \end{bmatrix} \times \begin{bmatrix} b_1 \\ b_2 \\ b_3 \end{bmatrix} = \begin{bmatrix} a_2 {\cdot} b_3 - a_3 {\cdot} b_2 \\ a_3 {\cdot} b_1 - a_1 {\cdot} b_3 \\ a_1 {\cdot} b_2 - a_2 {\cdot} b_1 \end{bmatrix}$ \\

			\text{nicht kommutativ!:} &	$\overrightarrow{a} \times \overrightarrow{b}$ & $= -(\overrightarrow{b} \times \overrightarrow{a})$ \\

			\text{Distributivgesetz:} &	$\overrightarrow{a} \times (\overrightarrow{b} + \overrightarrow{c})$ &= $\overrightarrow{a} \times \overrightarrow{b} + \overrightarrow{a} \times \overrightarrow{c}$ \\

			\text{Alt. Schreibweisen:} & $\overrightarrow{a} \times \overrightarrow{b}$ & $\equiv \overrightarrow{a} \wedge \overrightarrow{b} \equiv	[\overrightarrow{a}, \overrightarrow{b}]$
		\end{tabular}
		
	\end{emphbox}
	mit: $\overrightarrow{n}$: Einheitsvektor senkrecht zu $\overrightarrow{a}$ und $\overrightarrow{b}$
	Kreuzprodukt bildet ein Rechtssystem. Rechte-Hand-Regel: x: Daumen, y: Zeigefinger (abgespreizt), z: Mittelfinger

%	\begin{emphbox}
%	\begin{align*}
%		\text{mit: \overrightarrow{n}: Einheitsvektor senkrecht zu \overrightarrow{a} und \overrightarrow{b} }
%		\end{align*}
%	\end{emphbox}
	
\end{sectionbox}

\section{Elektrotechnik}
	
\begin{sectionbox}
	\subsection{Ohmsches Gesetz}

	\begin{emphbox}
	Ohmsches Gesestz: $ U = R \cdot I = \frac{I}{G} $
	\end{emphbox}

	\subsection{Leistung}

	\begin{emphbox}
	$ P = U \cdot I = \frac{U^2}{R} = I^2 \cdot R $
	\end{emphbox}
	

		
\end{sectionbox}

\begin{sectionbox}
	\subsection{Kirchoffsche Gesetze}

	Knotenregel:
	\begin{emphbox}
		In einem Knoten: Summe der zufliessenden Ströme = Summe der abfliessenden Ströme\\
		$\sum _{k=1}^{n}I_k = 0$
	\end{emphbox}
	
	Maschenregel:
	\begin{emphbox}
		In einer Masche: Summe der Teilspannungen addieren sich zu Null. Pfeilrichtungen beachten!\\
		$\sum _{k=1}^{n}U_k = 0$
	\end{emphbox}

\end{sectionbox}


\begin{sectionbox}
	\subsection{Widerstand}

	\begin{emphbox}
	Leitwert:	$ G = \frac{1}{R} $\\
	\end{emphbox}	
	
	Parallelschaltung:
	\begin{emphbox}
		\begin{align*}
			\frac{1}{R_{tot}} &= \sum _{k=1}^{n}\frac{1}{R_k} &= \frac{1}{R_{1}} + \frac{1}{R_{2}} + \ldots + \frac{1}{R_k}\\
			G_{tot} &= \sum _{k=1}^{n}G_k &= G_1 + G_2 + \ldots + G_k
		\end{align*}
	\end{emphbox}
	
	Serieschaltung:
	\begin{emphbox}
			\begin{align*}
			R_{tot} &= \sum _{k=1}^{n}R_k &= R_1 + R_2 + \ldots + R_k
		\end{align*}

	\end{emphbox}

	
\end{sectionbox}


\begin{sectionbox}
	\subsection{Kondensator}

	Lade-/Entladevorgang:
	\begin{emphbox}
		\begin{align*}
			u_C(t) &= U_0+ ΔU \cdot e^{-{\frac{t}{τ}}} = U_0+ (U_{C,t0}-U_0) \cdot e^{-{\frac{t}{τ}}} \\
			i_C(t) &= \frac{u_C(t)}{R_C} = \frac{U_0}{R_C}+\frac{ΔU}{R_C}\cdot e^{-{\frac{t}{τ}}}
		\end{align*}
	\end{emphbox}

	Ladevorgang:
	\begin{emphbox}
		\begin{align*}
			u_C(t) &= U_0 \cdot (1-e^{-{\frac{t}{τ}}}) \\
			i_C(t) &= I_0 \cdot e^{-{\frac{t}{τ}}}
		\end{align*}
	\end{emphbox}

	Entladevorgang:
	\begin{emphbox}
		\begin{align*}
			u_C(t) &= U_0 \cdot e^{-{\frac{t}{τ}}} \\
			i_C(t) &= -I_0 \cdot e^{-{\frac{t}{τ}}}
		\end{align*}
	\end{emphbox}

mit
\begin{emphbox}
	$I_0 = \frac{U_0}{R_C}$
	
	$τ = R_C \cdot C$
\end{emphbox}

	
\end{sectionbox}

\begin{sectionbox}
	\subsection{Spule}

Lade-/Entladevorgang:
\begin{emphbox}
	$u_C(t) = U_0+ ΔU \cdot e^{-{\frac{t}{τ}}} = U_0+ (U_{C,t0}-U_0) \cdot e^{-{\frac{t}{τ}}} ???$

	$i_C(t) = \frac{u_C(t)}{R_C} = \frac{U_0}{R_C}+\frac{ΔU}{R_C}\cdot e^{-{\frac{t}{τ}}} ???$
\end{emphbox}

Ladevorgang:
\begin{emphbox}

	$i_L(t) = I_0 \cdot (1-e^{-{\frac{t}{τ}}})$

	$u_L(t) = U_0 \cdot e^{-{\frac{t}{τ}}}$

\end{emphbox}

Entladevorgang:
\begin{emphbox}
	$u_L(t) = -U_0 \cdot e^{-{\frac{t}{τ}}}$

	$i_L(t) = I_0 \cdot e^{-{\frac{t}{τ}}}$
\end{emphbox}

mit
\begin{emphbox}
	$I_0 = \frac{U_0}{R_L}$
	
	$τ = \frac{L}{R_L}$	
\end{emphbox}

	
\end{sectionbox}
\section{Elektronik}

\subsection{Operationsverstärker}

\subsubsection{Allgemein}
	
	\resizebox{3cm}{!}{	
	\begin{circuitikz}[european]
		\draw (0, 5) node[op amp,yscale=-1] (opamp) {}
		(opamp.-) to[short,-o] ++(-0.5, 0) coordinate(N)
		(opamp.+) to[short,-o] ++(-1, 0) coordinate(P)
		(opamp.out) to[short,-o] ++(0.5, 0) coordinate(O)
		(O |- 0,4) node[rground](OG){} node[ocirc] {}
		(OG -| N) node[rground](ON){} node[ocirc]{}
		(OG -| P) node[rground](OP){} node[ocirc]{}
		% draw voltage arrows
		(P) to [open, v_=$u_+$, voltage=straight] (OP)
		(N) to [open, v^=$u_-$, voltage=straight] (ON)
		(O) to [open, v^=$u_{out}$, voltage=straight] (OG)
		;
	\end{circuitikz}
	} % resizebox

	\begin{tabular}{cc}
	\parbox{2.5cm}{
		\begin{emphbox}
			\begin{align*}
				u_{out} &= A_{OL} \cdot (u_+ - u_-)
			\end{align*}
		\end{emphbox}
	} & 
	
	\parbox{3cm}{
		\begin{symbolbox}
			$A_{OL}$: Open-loop gain
		\end{symbolbox}
	}
\end{tabular}

\subsubsection{Nichtinvertierender Verstärker (Elektrometerverstärker)}

	\resizebox{3cm}{!}{	
			\begin{circuitikz}[european]
	\draw (0, 5) node[op amp,yscale=-1] (op) {}
	(op.+) to[short,-o] ++(-1, 0) coordinate(P)
	(op.out) to[short,-o] ++(1, 0) coordinate(O)
	(op.out) node[circ] {} to[european resistor, l=$R_2$] ++(0, -2) coordinate(R)
	to[european resistor, l=$R_1$] ++(0, -2)
	node[rground](G){}
	(O |- G) node[rground](OG){} node[ocirc] {}
	(R) to[short,*-] (op.- |- R) to[short] (op.-)
	%		(OG -| N) node[rground](ON){} node[ocirc]{}
	(OG -| P) node[rground](OP){} node[ocirc]{}
	% draw voltage arrows
	(P) to [open, v_=$u_{in}$, voltage=straight] (OP)
	(O) to [open, v^=$u_{out}$, voltage=straight] (OG)
	;
\end{circuitikz}
	} % resizebox

	\begin{emphbox}
		\begin{align*}
			U_{out} &= (1 + \frac{R_2}{R_1}) \cdot U_{in}
		\end{align*}
	\end{emphbox}

\subsubsection{Spannungsfolger}

Spezialfall des nichtinv. Verstärkers mit $R_2 = 0\ohm$ und $R_1 = \infty\ohm$.


\subsubsection{Invertierender Addierer/Summierverstärker}

	\resizebox{6cm}{!}{
		\begin{tikzpicture}
	% Paths, nodes and wires:
	\coordinate (I3) at (0,7);
	\coordinate (I2) at (-1,8);
	\coordinate (I1) at (-2,9);
	\coordinate (GO) at (7,5);
	\draw node[op amp](op) at (4.2, 6.5) {};
	\draw (I1 -| op.+) to[european resistor, l=$R_f$] (I1 -| op.out) to[short,-*] (op.out);
	\draw (I1) to[short,o-] (I1 -| I3) to[european resistor, l=$R_1$] (I1 -| op.-);
	\draw (I2) to[short,o-] (I2 -| I3) to[european resistor, l=$R_2$] (I2 -| op.-);
	\draw (I3) to[european resistor, l=$R_3$] (I3 -| op.-);
	\draw (3, 8) to[short,-*] (3, 9);
	\draw (3, 7) to[short,*-*] (3, 8);
	
	\draw (I3) node[ocirc] {};
	
	\draw (op.out) to[short,-o] (op.out -| GO) to[open,v^=$U_{out}$, voltage=straight] (GO) node[rground] {};
	
	\draw (5.4, 6.5) -| (6, 6.5);
	\draw (op.+) -- (op.+ |- GO) node[rground]{};
	
	% draw voltage arrows and ground symbols, aligned to GO
	\draw (I3) to [open, v^=$U_3$, voltage=straight] (GO -| I3)
	node[rground] {} node[ocirc] {};
	\draw (I2) to [open, v^=$U_2$, voltage=straight] (GO -| I2)
	node[rground] {} node[ocirc] {};
	\draw (I1) to [open, v^=$U_1$, voltage=straight] (GO -| I1)
	node[rground] {} node[ocirc] {};
\end{tikzpicture}
	}
	
	\begin{emphbox}
		\begin{align*}
			U_{out} &= -R_f \cdot (\frac{U_1}{R_1} + \frac{U_2}{R_2} + \frac{U_3}{R_3})
		\end{align*}
	\end{emphbox}

\subsubsection{Invertierender Verstärker}

Der invertierende Verstärker ist ein Spezialfall des inv. Addierers mit einem einzelnen Eingang.

\subsubsection{Differenzverstärker/Subtrahierverstärker}

	\resizebox{6cm}{!}{
		\begin{circuitikz}
	% Paths, nodes and wires:
	
	\draw node[op amp](op) at (4.2, 6.5) {};
	\draw (op.-) to[short,*-] ++(0,1) coordinate(N) to[european resistor,l=$R_f$] (N-|op.out) to[short,-*] (op.out);
	
	\draw (op.+) node[circ]{} to[european resistor,l=$R_g$] ++(0, -2) coordinate(G) node[rground] {};
	
	\draw (op.+) to[european resistor, l=$R_2$] ++(-2, 0) coordinate(I2) node[ocirc]{} to[open,v^=$U_2$, voltage=straight] (G-|I2) node[ocirc]{} node[rground] {};
	
	\draw (op.-) to[european resistor, l=$R_1$]  ++(-2, 0) to[short] ++(-0.5, 0) coordinate(I1) node[ocirc]{} to[open,v^=$U_1$, voltage=straight] (G-|I1) node[ocirc]{} node[rground] {};
	
	\draw (op.out) to[short,-o] ++(0.5,0) coordinate(O) to[open,v^=$U_{out}$, voltage=straight] (O|-G) node[ocirc]{} node[rground] {};
	
\end{circuitikz}
	}

	\begin{emphbox}
		\begin{align*}
			not checked!
%			U_{out} &= U_2 \cdot \frac{R_1+R_f}{R_1} \cdot \frac{R_g}{R_2+R_g} - U_1 \frac{R_f}{R_1}
		\end{align*}
	\end{emphbox}

\subsubsection{Instrumentenverstärker}
	\resizebox{6cm}{!}{
		\begin{circuitikz}
	% Paths, nodes and wires:
	
	\draw node[op amp,yscale=-1](op1) at (1.2, 6.5) {};
	\draw node[op amp](op2) at (1.2, 0.5) {};
	\draw node[op amp](op3) at (5.6, 3.5) {};
	\draw (op1.out) to[european resistor, l=$R_1$] ++(0,-2) coordinate (N1) -| (op1.-);
	\draw (N1) node[circ]{} to[european resistor, l=$R_g$]  ++(0,-2) coordinate (N2) node[circ]{};
	\draw (op2.out) to[european resistor, l=$R_1$] ++(0,2) (N2) -| (op2.-);
	\draw (op1.out) to[european resistor, l=$R_2$, *-] ++(2,0) coordinate (N3) -- (op3.-);
	\draw (N3) to[european resistor, l=$R_3$, *-] ++(2,0) -| (op3.out) to[short,*-] ++(0.5,0) node[ocirc]{} to[open,v^=$U_{out}$, voltage=straight] ++(0,-5) coordinate(GND) node[ocirc]{} node[rground] {};
	\draw (op2.out) to[european resistor, l=$R_2$, *-] ++(2,0) coordinate (N4) -- (op3.+);
	
	\draw (op1.+) -- ++(-0.5,0) coordinate(I1) node[ocirc]{} to[open,v^=$U_1$, voltage=straight] ++(0,-2) node[ocirc]{} node[rground] {};
	
	\draw (op2.+) -- ++(-0.5,0) coordinate(I2) node[ocirc]{} to[open,v^=$U_2$, voltage=straight] ++(0,-2) node[ocirc]{} node[rground] {};
	
	\draw (N4) to[european resistor, l=$R_3$, *-] ++(2,0) coordinate(N5) -- (N5 |- GND) node[rground]{};
\end{circuitikz}
	}
	\begin{emphbox}
		\begin{align*}
			U_{out} &= (1+\frac{2R}{Rg}) \cdot \frac{R_3}{R_2} \cdot (U_2-U_1)
		\end{align*}
	\end{emphbox}
\subsubsection{Integrierer}
\subsubsection{Differenzierer}
\subsubsection{Tiefpass 1. Ordnung}

	\begin{circuitikz}
		% Paths, nodes and wires:
		\coordinate (I) at (0, 1);
		\draw (I) to[european resistor, l=$R$, o-*] ++(2,0) coordinate(N) -- ++(1,0) coordinate(O) node[ocirc]{};
		\draw (N) to[capacitor] ++(0,-2) coordinate(N2) node[ocirc]{};
		\draw (I|-N2) to[short,o-o] (O|-N2);
		\draw (I) to[open,v^=$U_1$, voltage=straight] (I|-N2);
		\draw (O) to[open,v^=$U_2$, voltage=straight] (O|-N2);
	\end{circuitikz}

\section{Trigonometrie}

\begin{sectionbox}
	
	\subsection{Allgemeines Dreieck}
		Winkelsumme im Dreick
		\begin{emphbox}
			Winkelsumme: $ \alpha + \beta + \gamma = 180° $
		\end{emphbox}
	\begin{tikzpicture}

		% Define the points of the triangle
		\coordinate (A) at (0,0);
		\coordinate (B) at (4,0);
		\coordinate (C) at (3.0,2.0);
		
		% Draw the triangle
		\draw [thick] (A) -- (B) -- (C) -- cycle;
		
		% Mark the angles
		\centerarc[](A)(0.0:33.7:0.4)
		\node[above right= 0.1cm and 0.5cm of A] {$\alpha$};

		\centerarc[](B)(180.0:116.6:0.4)
		\node[above left= 0.1cm and 0.5cm of B] {$\beta$};

		\centerarc[](C)(213.7:296.6:0.4)
		\node[below left= 0.5cm and 0.1cm of C] {$\gamma$};

		% Label the vertices
		\node[below left] at (A) {A};
		\node[above right] at (C) {C};
		\node[below right] at (B) {B};

		% Label the sides
		\node (a) at ($(B)!0.5!(C)$) [anchor=west] {$a$};
		\node (b) at ($(A)!0.5!(C)$) [anchor=south east] {$b$};
		\node (c) at ($(A)!0.5!(B)$) [anchor=north] {$c$};
		
	\end{tikzpicture}
	
	\subsection{Rechtwinkliges Dreieck}
	
	a: Ankathete von $\beta$, Gegenkathete von $\alpha$\\
	b: Ankathete von $\alpha$, Gegenkathete von $\beta$\\
	c: Hypotenuse \\
	
	\begin{tikzpicture}
		
		% Define the points A, B (endpoints of the diameter)
		\coordinate (A) at (0, 0);
		\coordinate (B) at (5, 0);
		
		% Define the midpoint M
		\coordinate (M) at ($(A)!0.5!(B)$);
		
		% Define point C on the semicircle
		\coordinate (C) at ($(M) + (50:2.5cm)$);  % Angle 50 degrees from the horizontal
		
		% Draw the semicircle
		\draw[] (A) arc[start angle=180, end angle=0, radius=2.5cm];
		
		% Draw the diameter
		\draw[thick] (A) -- (B);
		
		% Draw right-angled triangle
		\draw[thick] (A) -- (C) -- (B) -- cycle;

		% Mark the points A, B, C
		\fill[black] (A) circle (2pt) node[below left] {$A$};
		\fill[black] (B) circle (2pt) node[below right] {$B$};
		\fill[black] (C) circle (2pt) node[above right] {$C$};

		% Draw angle arcs and labels
		\centerarc[](A)(0:25:0.4)
		\node[above right= 0.1cm and 0.5cm of A] {$\alpha$};

		\centerarc[](B)(180:115:0.4)
		\node[above left= 0.1cm and 0.5cm of B] {$\beta$};

		% Right angle dot
		\centerarc[](C)(205:295:0.4)
		\draw[thick,fill=black] ($(C) + (250:0.2)$) circle[radius=0.05];

		\node at ($(A)!0.5!(C)$) [anchor=south east] {$b$};
		\node at ($(B)!0.5!(C)$) [anchor=west] {$a$};
		\node at ($(A)!0.5!(B)$) [below] {$c$};
		
	\end{tikzpicture}

	\subsection{Sinus und Cosinus}
		Der Sinus ist eine ungerade Funktion \\
		Der Cosinus ist eine gerade Funktion
		\begin{emphbox}
			\begin{align*}
				\sin(\alpha) &= \frac{a}{c} = \cos(\beta) \\
				\sin(\beta) &= \frac{b}{c} = \cos(\alpha)
			\end{align*}
		\end{emphbox}

	\subsection{Symmetrieeigenschaften}
		\begin{emphbox}
			\begin{align*}
			\sin (-\alpha) &= -\sin(\alpha) \\
			\sin (90 \degree + \alpha) &= \sin (90 \degree - \alpha) \\
			\cos (-\alpha) &= \cos(\alpha)\\
			\cos (90 \degree + \alpha) &= -\cos (90 \degree - \alpha)
			\end{align*}
		\end{emphbox}


	\subsection{Tangens}
		\begin{emphbox}
			\begin{align*}
			\tan(\alpha) &= \frac{a}{b} = \frac{\sin(\alpha)}{\cos(\alpha)} \\
			\tan(\beta) &= \frac{b}{a} = \frac{\sin(\beta)}{\cos(\beta)}
			\end{align*}
		\end{emphbox}
		
	\subsection{Cotangens}
		\begin{emphbox}
			\begin{align*}
			\cot(\alpha) &= \frac{1}{\tan(\alpha)} \\
			\cot \alpha &= \frac{\cos \alpha}{\sin \alpha} 		
			\end{align*}
		\end{emphbox}

	\subsection{Trigonometrischer Pythagoras}
		\begin{emphbox}
			$ \sin ^2 \alpha + \cos ^2 \alpha = 1 $
		\end{emphbox}
		
	\subsection{Sinussatz}
		\begin{emphbox}
			$ \frac{a}{\sin \alpha} = \frac{b}{\sin \beta} = \frac{c}{\sin \gamma} = \frac{a \cdot b \cdot c}{2 \cdot A} = 2 \cdot R$
		\end{emphbox}

	\subsection{Cosinussatz}
		\begin{emphbox}
			$ c^2 = a^2 + b^2 - 2 \cdot a \cdot b \cdot \cos \gamma $
		\end{emphbox}

\begin{symbolbox}
	A = Fläche\\
	R = Radius des Umkreises
\end{symbolbox}

\begin{bluebox}
	Test
\end{bluebox}


\end{sectionbox}


\begin{sectionbox}
	\subsection{Cosinus}

	Text goes here ...


\end{sectionbox}



\section{Akustik}

\begin{sectionbox}
	\subsection{Mach}

\begin{tabular}{cc}

	\parbox{2.5cm}{
		\begin{emphbox}
			\begin{align*}
				\sin \mu &= \frac{c}{v} = \frac{1}{M}\\
				\mu &= \arcsin \frac{1}{M}
			\end{align*}
		\end{emphbox}
	} & 
	\parbox{3cm}{


		\begin{symbolbox}
			c: Schallgeschwindigkeit \\
			v: Objektgeschwindigkeit \\
			M: Machzahl \\
			$\mu$: Machwinkel
		\end{symbolbox}
	}
\end{tabular}

	\resizebox{5cm}{!}{
		
		\def\dotMarkRightAngle[size=#1](#2,#3,#4){%
			\draw ($(#3)!#1!(#2)$) -- 
			($($(#3)!#1!(#2)$)!#1!90:(#2)$) --
			($(#3)!#1!(#4)$);
			\path (#3) --node[circle,fill,inner sep=.5pt]{} ($($(#3)!#1!(#2)$)!#1!90:(#2)$);
		}
		
		\begin{tikzpicture}
			
			\coordinate (OO) at (0,0);				
			
			% Circle centers
			\coordinate (P3) at (1,0);				
			\coordinate (P2) at (2,0);				
			\coordinate (P1) at (3,0);				
			
			% Tangential points
			\coordinate (S3) at ({1 *(1-sin(30)*sin(30)},{1*sin(30)*cos(30)});				
			\coordinate (S2) at ({2 *(1-sin(30)*sin(30)},{2*sin(30)*cos(30)});				
			\coordinate (S1) at ({3 *(1-sin(30)*sin(30)},{3*sin(30)*cos(30)});				
			
			% Endpoint of Mach front
			\coordinate (SX) at (4,{4*tan(30)});				
			
			\draw [-Stealth,dashed] (P3) -- node[above] {} (S3);
			\draw [-Stealth,dashed] (P2) -- node[above] {} (S2);
			\draw [-Stealth] (P1) -- node[right] {Schallgeschwindigkeit c} (S1);
			
			\coordinate (O) at (4,0);
			%		\coordinate (C) at (0,3);
			%		\coordinate (D) at ($(C)!(P3)!(B)$);
			
			\draw [-Stealth] (O) -- node[below] {Objektgeschwindigkeit v} (OO);
			\draw [dashed] (OO) -- node[left] {Machwelle} (SX);
			
			\pic [draw, angle radius=3mm, angle eccentricity=1.2mm] {angle = P3--OO--S3};
			\node[above = 0.1cm and 0.8cm of OO] {$\mu$};
			
			\dotMarkRightAngle[size=6pt](OO,S1,P1);
			
			\draw (P3) [dashed] circle ({1*sin(30)});
			\draw (P2) [dashed] circle ({2*sin(30)});
			\draw (P1) circle ({3*sin(30)});
			
			
		\end{tikzpicture}
		
	}		
	
	
	$M < 1.0$: Subsonic \\
	$M = 1.0$: Transonic \\
	$M > 1.0$: Supersonic \\
	$M > 5.0$: Hypersonic
	
\end{sectionbox}


% ======================================================================
% End
% ======================================================================
\end{document}
