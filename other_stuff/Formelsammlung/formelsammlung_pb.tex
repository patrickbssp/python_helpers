% % % % % % % % % % % % % % % % % % % % % % % % % % % % % % % % % % % % % % % %
% LaTeX4EI Template for Cheat Sheets                                Version 1.1
%
% Authors: Emanuel Regnath, Martin Zellner
% Contact: info@latex4ei.de
% Encode: UTF-8, tabwidth = 4, newline = LF
% % % % % % % % % % % % % % % % % % % % % % % % % % % % % % % % % % % % % % % %


% ======================================================================
% Document Settings
% ======================================================================

% possible options: color/nocolor, english/german, threecolumn
% defaults: color, english
\documentclass[english]{latex4ei/latex4ei_sheet}
\usepackage{empheq}
\usepackage{gensymb}
\usepackage{tikz}


% set document information
\title{LaTeX4EI \\ Cheat Sheet}
\author{LaTeX4EI}					% optional, delete if unchanged
\myemail{info@latex4ei.de}			% optional, delete if unchanged
\mywebsite{www.latex4ei.de}			% optional, delete if unchanged


% ======================================================================
% Begin
% ======================================================================
\begin{document}

% Title
% ----------------------------------------------------------------------
\maketitle   % requires ./img/Logo.pdf

% Section
% ----------------------------------------------------------------------
\section{Elektrotechnik}
	
\begin{sectionbox}
	\subsection{Ohmsches Gesetz}

	\begin{emphbox}
	Ohmsches Gesestz: $ U = R \cdot I = \frac{I}{G} $
	\end{emphbox}

	\subsection{Leistung}

	\begin{emphbox}
	$ P = U \cdot I = \frac{U^2}{R} = I^2 \cdot R $
	\end{emphbox}
	

		
\end{sectionbox}

\begin{sectionbox}
	\subsection{Kirchoffsche Gesetze}

	Knotenregel:
	\begin{emphbox}
		In einem Knoten: Summe der zufliessenden Ströme = Summe der abfliessenden Ströme\\
		$\sum _{k=1}^{n}I_k = 0$
	\end{emphbox}
	
	Maschenregel:
	\begin{emphbox}
		In einer Masche: Summe der Teilspannungen addieren sich zu Null. Pfeilrichtungen beachten!\\
		$\sum _{k=1}^{n}U_k = 0$
	\end{emphbox}

\end{sectionbox}


\begin{sectionbox}
	\subsection{Widerstand}

	\begin{emphbox}
	Leitwert:	$ G = \frac{1}{R} $\\
	\end{emphbox}	
	
	Parallelschaltung:
	\begin{emphbox}
		\begin{align*}
			\frac{1}{R_{tot}} &= \sum _{k=1}^{n}\frac{1}{R_k} &= \frac{1}{R_{1}} + \frac{1}{R_{2}} + \ldots + \frac{1}{R_k}\\
			G_{tot} &= \sum _{k=1}^{n}G_k &= G_1 + G_2 + \ldots + G_k
		\end{align*}
	\end{emphbox}
	
	Serieschaltung:
	\begin{emphbox}
			\begin{align*}
			R_{tot} &= \sum _{k=1}^{n}R_k &= R_1 + R_2 + \ldots + R_k
		\end{align*}

	\end{emphbox}

	
\end{sectionbox}


\begin{sectionbox}
	\subsection{Kondensator}

	Lade-/Entladevorgang:
	\begin{emphbox}
		\begin{align*}
			u_C(t) &= U_0+ ΔU \cdot e^{-{\frac{t}{τ}}} = U_0+ (U_{C,t0}-U_0) \cdot e^{-{\frac{t}{τ}}} \\
			i_C(t) &= \frac{u_C(t)}{R_C} = \frac{U_0}{R_C}+\frac{ΔU}{R_C}\cdot e^{-{\frac{t}{τ}}}
		\end{align*}
	\end{emphbox}

	Ladevorgang:
	\begin{emphbox}
		\begin{align*}
			u_C(t) &= U_0 \cdot (1-e^{-{\frac{t}{τ}}}) \\
			i_C(t) &= I_0 \cdot e^{-{\frac{t}{τ}}}
		\end{align*}
	\end{emphbox}

	Entladevorgang:
	\begin{emphbox}
		\begin{align*}
			u_C(t) &= U_0 \cdot e^{-{\frac{t}{τ}}} \\
			i_C(t) &= -I_0 \cdot e^{-{\frac{t}{τ}}}
		\end{align*}
	\end{emphbox}

mit
\begin{emphbox}
	$I_0 = \frac{U_0}{R_C}$
	
	$τ = R_C \cdot C$
\end{emphbox}

	
\end{sectionbox}

\begin{sectionbox}
	\subsection{Spule}

Lade-/Entladevorgang:
\begin{emphbox}
	$u_C(t) = U_0+ ΔU \cdot e^{-{\frac{t}{τ}}} = U_0+ (U_{C,t0}-U_0) \cdot e^{-{\frac{t}{τ}}} ???$

	$i_C(t) = \frac{u_C(t)}{R_C} = \frac{U_0}{R_C}+\frac{ΔU}{R_C}\cdot e^{-{\frac{t}{τ}}} ???$
\end{emphbox}

Ladevorgang:
\begin{emphbox}

	$i_L(t) = I_0 \cdot (1-e^{-{\frac{t}{τ}}})$

	$u_L(t) = U_0 \cdot e^{-{\frac{t}{τ}}}$

\end{emphbox}

Entladevorgang:
\begin{emphbox}
	$u_L(t) = -U_0 \cdot e^{-{\frac{t}{τ}}}$

	$i_L(t) = I_0 \cdot e^{-{\frac{t}{τ}}}$
\end{emphbox}

mit
\begin{emphbox}
	$I_0 = \frac{U_0}{R_L}$
	
	$τ = \frac{L}{R_L}$	
\end{emphbox}

	
\end{sectionbox}


\section{Trigonometrie}

\begin{sectionbox}
	\subsection{Sinus}
		\begin{emphbox}
			\begin{align*}
			 \sin \alpha &= \frac{Gegenkathete}{Hypotenuse} \\
			 \sin (-\alpha) &= -\sin \alpha \\
			 \sin (90 \degree + \alpha) &= \sin (90 \degree - \alpha)
			\end{align*}
		\end{emphbox}
		ungerade Funktion
	\subsection{Cosinus}
		\begin{emphbox}
			\begin{align*}
			\cos \alpha &= \frac{Ankathete}{Hypotenuse}\\
			\cos (-\alpha) &= \cos \alpha\\
			\cos (90 \degree + \alpha) &= -\cos (90 \degree - \alpha)
			\end{align*}
		\end{emphbox}
		Gerade Funktion

	\subsection{Tangens}
		\begin{emphbox}
			\begin{align*}
			\tan \alpha &= \frac{Gegenkathete}{Ankathete} \\
			\tan \alpha &= \frac{\sin \alpha}{\cos \alpha} 
			\end{align*}
		\end{emphbox}
		
	\subsection{Cotangens}
		\begin{emphbox}
			\begin{align*}
			\cot \alpha &= \frac{Ankathete}{Gegenkathete} \\
			\cot \alpha &= \frac{\cos \alpha}{\sin \alpha} 		
			\end{align*}
		\end{emphbox}
	mit:\\
	a = Gegenkathete\\
	b = Ankathete\\
	c = Hypotenuse

	\begin{tikzpicture}
		\draw (0, 0) -- (3, 0) -- (2, 2) -- (0, 0);
		\coordinate[label=left:$A$] (A) at (0, 0);
		\coordinate[label=right:$B$] (B) at (3, 0);
		\coordinate[label=above:$C$] (C) at (2, 2);
	\end{tikzpicture}

	\subsection{Trigonometrischer Pythagoras}
		\begin{emphbox}
			$ \sin ^2 \alpha + \cos ^2 \alpha = 1 $
		\end{emphbox}
		
	\subsection{Sinussatz}
		\begin{emphbox}
			$ \frac{a}{\sin \alpha} = \frac{b}{\sin \beta} = \frac{c}{\sin \gamma} = \frac{a \cdot b \cdot c}{2 \cdot A} = 2 \cdot R$
		\end{emphbox}

	\subsection{Cosinussatz}
		\begin{emphbox}
			$ c^2 = a^2 + b^2 - 2 \cdot a \cdot b \cdot \cos \gamma$
		\end{emphbox}

\begin{symbolbox}
	A = Fläche\\
	R = Radius des Umkreises
\end{symbolbox}

\begin{bluebox}
	Test
\end{bluebox}


\end{sectionbox}


\begin{sectionbox}
	\subsection{Cosinus}

	Text goes here ...


\end{sectionbox}





% ======================================================================
% End
% ======================================================================
\end{document}
